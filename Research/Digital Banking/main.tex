\documentclass{article}
\usepackage[utf8]{inputenc}
\usepackage{url}


\title{Digital Banking}
\author{Lesego Mabe - Le Bakeryto}
\date{February 2019}

\begin{document}

\maketitle

\section{Ways to Win in SA}

The South African Reserve Bank (SARB) is the local bank regulator that has created a friendly regulatory environment that encourages new digital banking players to entire and thrive in the market.\cite{Website:1} This may be a great access point for regulatory information we may need for our project.

Digital banking players will need to identify a clear target market of which it will center its services around.\cite{Website:1}. They also ought to use customer data insights from the knowledge they have gained by processing their customer's transactions in order to identify areas of improvement, and work on introducing a ecosystem of products and services to take advantage of the intelligence.\cite{Website:1}. 

\section{The Digital Trend}

The trend here seems to be that banks are moving to relying more on their electronic presence then physical presence in offices and branches. More business processes are becoming paperless and can be done in their respective applications. BankZero takes this to whole new level with all all of their customer services being app driven, that is to say, all of the bank's functionality is driven by the application.\cite{Website:2} This advantages BankZero as it is able to reduce costs to the benefit of customers.

This new drive is also looking to reduce the costs of operation and is eyeing new technologies and innovative ideas to achieve this.\cite{Website:1} Moving digital however also means that these banks should take care of not just physical security (bank vaults, security guards, etc) but digital security to secure their clients' resources. It would make sense for this kind of security to also be in the forefront of their thinking.

Gartner still indicates that Blockchain, a type of distributed ledger and data structure, is set to make a positive impact in the financial sector with promises of greater transparency, reduction of transaction settlement times, among other benefits.\cite{Website:3} Digital ethics and privacy is a point of focus for individuals, organisations and government with the increase over concerns of how information is used and where it is sent, and as such, the project needs to be cognizant of this at all times to avoid unnecessary backlash.\cite{Website:3}. The relevant laws and regulations ought to be consulted as well. 

\bibliographystyle{plain}
\bibliography{references}

\end{document}
